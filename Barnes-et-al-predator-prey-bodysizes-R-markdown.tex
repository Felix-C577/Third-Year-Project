% Options for packages loaded elsewhere
\PassOptionsToPackage{unicode}{hyperref}
\PassOptionsToPackage{hyphens}{url}
%
\documentclass[
]{article}
\usepackage{lmodern}
\usepackage{amssymb,amsmath}
\usepackage{ifxetex,ifluatex}
\ifnum 0\ifxetex 1\fi\ifluatex 1\fi=0 % if pdftex
  \usepackage[T1]{fontenc}
  \usepackage[utf8]{inputenc}
  \usepackage{textcomp} % provide euro and other symbols
\else % if luatex or xetex
  \usepackage{unicode-math}
  \defaultfontfeatures{Scale=MatchLowercase}
  \defaultfontfeatures[\rmfamily]{Ligatures=TeX,Scale=1}
\fi
% Use upquote if available, for straight quotes in verbatim environments
\IfFileExists{upquote.sty}{\usepackage{upquote}}{}
\IfFileExists{microtype.sty}{% use microtype if available
  \usepackage[]{microtype}
  \UseMicrotypeSet[protrusion]{basicmath} % disable protrusion for tt fonts
}{}
\makeatletter
\@ifundefined{KOMAClassName}{% if non-KOMA class
  \IfFileExists{parskip.sty}{%
    \usepackage{parskip}
  }{% else
    \setlength{\parindent}{0pt}
    \setlength{\parskip}{6pt plus 2pt minus 1pt}}
}{% if KOMA class
  \KOMAoptions{parskip=half}}
\makeatother
\usepackage{xcolor}
\IfFileExists{xurl.sty}{\usepackage{xurl}}{} % add URL line breaks if available
\IfFileExists{bookmark.sty}{\usepackage{bookmark}}{\usepackage{hyperref}}
\hypersetup{
  pdftitle={Barnes et al: Predator and Prey Bodysizes in Marine Food Webs},
  pdfauthor={Felix Carter},
  hidelinks,
  pdfcreator={LaTeX via pandoc}}
\urlstyle{same} % disable monospaced font for URLs
\usepackage[margin=1in]{geometry}
\usepackage{color}
\usepackage{fancyvrb}
\newcommand{\VerbBar}{|}
\newcommand{\VERB}{\Verb[commandchars=\\\{\}]}
\DefineVerbatimEnvironment{Highlighting}{Verbatim}{commandchars=\\\{\}}
% Add ',fontsize=\small' for more characters per line
\usepackage{framed}
\definecolor{shadecolor}{RGB}{248,248,248}
\newenvironment{Shaded}{\begin{snugshade}}{\end{snugshade}}
\newcommand{\AlertTok}[1]{\textcolor[rgb]{0.94,0.16,0.16}{#1}}
\newcommand{\AnnotationTok}[1]{\textcolor[rgb]{0.56,0.35,0.01}{\textbf{\textit{#1}}}}
\newcommand{\AttributeTok}[1]{\textcolor[rgb]{0.77,0.63,0.00}{#1}}
\newcommand{\BaseNTok}[1]{\textcolor[rgb]{0.00,0.00,0.81}{#1}}
\newcommand{\BuiltInTok}[1]{#1}
\newcommand{\CharTok}[1]{\textcolor[rgb]{0.31,0.60,0.02}{#1}}
\newcommand{\CommentTok}[1]{\textcolor[rgb]{0.56,0.35,0.01}{\textit{#1}}}
\newcommand{\CommentVarTok}[1]{\textcolor[rgb]{0.56,0.35,0.01}{\textbf{\textit{#1}}}}
\newcommand{\ConstantTok}[1]{\textcolor[rgb]{0.00,0.00,0.00}{#1}}
\newcommand{\ControlFlowTok}[1]{\textcolor[rgb]{0.13,0.29,0.53}{\textbf{#1}}}
\newcommand{\DataTypeTok}[1]{\textcolor[rgb]{0.13,0.29,0.53}{#1}}
\newcommand{\DecValTok}[1]{\textcolor[rgb]{0.00,0.00,0.81}{#1}}
\newcommand{\DocumentationTok}[1]{\textcolor[rgb]{0.56,0.35,0.01}{\textbf{\textit{#1}}}}
\newcommand{\ErrorTok}[1]{\textcolor[rgb]{0.64,0.00,0.00}{\textbf{#1}}}
\newcommand{\ExtensionTok}[1]{#1}
\newcommand{\FloatTok}[1]{\textcolor[rgb]{0.00,0.00,0.81}{#1}}
\newcommand{\FunctionTok}[1]{\textcolor[rgb]{0.00,0.00,0.00}{#1}}
\newcommand{\ImportTok}[1]{#1}
\newcommand{\InformationTok}[1]{\textcolor[rgb]{0.56,0.35,0.01}{\textbf{\textit{#1}}}}
\newcommand{\KeywordTok}[1]{\textcolor[rgb]{0.13,0.29,0.53}{\textbf{#1}}}
\newcommand{\NormalTok}[1]{#1}
\newcommand{\OperatorTok}[1]{\textcolor[rgb]{0.81,0.36,0.00}{\textbf{#1}}}
\newcommand{\OtherTok}[1]{\textcolor[rgb]{0.56,0.35,0.01}{#1}}
\newcommand{\PreprocessorTok}[1]{\textcolor[rgb]{0.56,0.35,0.01}{\textit{#1}}}
\newcommand{\RegionMarkerTok}[1]{#1}
\newcommand{\SpecialCharTok}[1]{\textcolor[rgb]{0.00,0.00,0.00}{#1}}
\newcommand{\SpecialStringTok}[1]{\textcolor[rgb]{0.31,0.60,0.02}{#1}}
\newcommand{\StringTok}[1]{\textcolor[rgb]{0.31,0.60,0.02}{#1}}
\newcommand{\VariableTok}[1]{\textcolor[rgb]{0.00,0.00,0.00}{#1}}
\newcommand{\VerbatimStringTok}[1]{\textcolor[rgb]{0.31,0.60,0.02}{#1}}
\newcommand{\WarningTok}[1]{\textcolor[rgb]{0.56,0.35,0.01}{\textbf{\textit{#1}}}}
\usepackage{graphicx,grffile}
\makeatletter
\def\maxwidth{\ifdim\Gin@nat@width>\linewidth\linewidth\else\Gin@nat@width\fi}
\def\maxheight{\ifdim\Gin@nat@height>\textheight\textheight\else\Gin@nat@height\fi}
\makeatother
% Scale images if necessary, so that they will not overflow the page
% margins by default, and it is still possible to overwrite the defaults
% using explicit options in \includegraphics[width, height, ...]{}
\setkeys{Gin}{width=\maxwidth,height=\maxheight,keepaspectratio}
% Set default figure placement to htbp
\makeatletter
\def\fps@figure{htbp}
\makeatother
\setlength{\emergencystretch}{3em} % prevent overfull lines
\providecommand{\tightlist}{%
  \setlength{\itemsep}{0pt}\setlength{\parskip}{0pt}}
\setcounter{secnumdepth}{-\maxdimen} % remove section numbering

\title{Barnes et al: Predator and Prey Bodysizes in Marine Food Webs}
\author{Felix Carter}
\date{27/10/2020}

\begin{document}
\maketitle

\hypertarget{background-information}{%
\subsection{Background Information}\label{background-information}}

The abstract of this paper explains that we need information of the
relationships between predator and prey size in order to understand how
species and size classes interact within food webs. This data contains
almost 35,000 records from a range of environments and locations, for a
vast range of predator sizes - from 0.1 milligrams to 415 kilograms, and
prey sizes from 75 picograms to over 4.5 kilograms. Information about
the species such as location, scientific name, life stage and habitat
description is also included.

\hypertarget{importing-the-data}{%
\subsection{Importing the Data}\label{importing-the-data}}

Firstly, I had to download the text file containing the data from the
Barnes et al study(the data can be found at
\url{https://figshare.com/articles/dataset/Full_Archive/3529112}). Then
I converted it to a .csv file in Excel. In this code, we import the
library tidyverse to use its features (including ggplot) in the future.
Then we import the data file into R and save it to the pred\_prey\_data
dataframe.

\begin{Shaded}
\begin{Highlighting}[]
\KeywordTok{library}\NormalTok{(tidyverse)}
\end{Highlighting}
\end{Shaded}

\begin{verbatim}
## -- Attaching packages ------------------------------------------------------- tidyverse 1.3.0 --
\end{verbatim}

\begin{verbatim}
## v ggplot2 3.3.2     v purrr   0.3.4
## v tibble  3.0.3     v dplyr   1.0.1
## v tidyr   1.1.1     v stringr 1.4.0
## v readr   1.3.1     v forcats 0.5.0
\end{verbatim}

\begin{verbatim}
## -- Conflicts ---------------------------------------------------------- tidyverse_conflicts() --
## x dplyr::filter() masks stats::filter()
## x dplyr::lag()    masks stats::lag()
\end{verbatim}

\begin{Shaded}
\begin{Highlighting}[]
\NormalTok{pred_prey_data <-}\StringTok{ }\KeywordTok{read.csv}\NormalTok{(}\StringTok{"Predator_and_prey_body_sizes_in_marine_food_webs_vsn4_csvcopy.csv"}\NormalTok{)}
\end{Highlighting}
\end{Shaded}

\hypertarget{starting-to-look-at-the-data}{%
\subsection{Starting to look at the
data}\label{starting-to-look-at-the-data}}

Plotting masses of the prey masses shows the variation in the data.

\begin{Shaded}
\begin{Highlighting}[]
\KeywordTok{plot}\NormalTok{(pred_prey_data}\OperatorTok{$}\NormalTok{Prey.mass)}
\end{Highlighting}
\end{Shaded}

\includegraphics{Barnes-et-al-predator-prey-bodysizes-R-markdown_files/figure-latex/unnamed-chunk-3-1.pdf}

This graph shows that most of the data entries are under 1000 units for
Prey.mass, however some of them are over 4000 units. We can view the
units for this variable:

\begin{Shaded}
\begin{Highlighting}[]
\KeywordTok{unique}\NormalTok{(pred_prey_data}\OperatorTok{$}\NormalTok{Prey.mass.unit)}
\end{Highlighting}
\end{Shaded}

\begin{verbatim}
## [1] "g"  "mg"
\end{verbatim}

\begin{Shaded}
\begin{Highlighting}[]
\KeywordTok{unique}\NormalTok{(pred_prey_data}\OperatorTok{$}\NormalTok{Predator.mass.unit)}
\end{Highlighting}
\end{Shaded}

\begin{verbatim}
## [1] "g"
\end{verbatim}

Which shows that for Prey Mass as in the graph above, some masses are
measured in milligrams but some are measured in grams. Fortunately, for
Predator Mass all units are in grams. Ideally we would like to get all
measurements in the same unit (grams) for fair comparison in graphs. In
the code below, we multiply any Prey Mass measurement that is in
milligrams by 0.001 to convert it to grams. We then set the mass unit
value to grams for these to ``g'' so this is accurate.

\begin{Shaded}
\begin{Highlighting}[]
\NormalTok{pred_prey_data}\OperatorTok{$}\NormalTok{Prey.mass[pred_prey_data}\OperatorTok{$}\NormalTok{Prey.mass.unit }\OperatorTok{==}\StringTok{ "mg"}\NormalTok{] <-}\StringTok{ }\FloatTok{0.001}\OperatorTok{*}\NormalTok{pred_prey_data}\OperatorTok{$}\NormalTok{Prey.mass[pred_prey_data}\OperatorTok{$}\NormalTok{Prey.mass.unit }\OperatorTok{==}\StringTok{ "mg"}\NormalTok{]}
\NormalTok{pred_prey_data}\OperatorTok{$}\NormalTok{Prey.mass.unit[pred_prey_data}\OperatorTok{$}\NormalTok{Prey.mass.unit }\OperatorTok{==}\StringTok{ "mg"}\NormalTok{] <-}\StringTok{ "g"}
\end{Highlighting}
\end{Shaded}

We can now see that all measurements for Prey mass are in grams:

\begin{Shaded}
\begin{Highlighting}[]
\KeywordTok{unique}\NormalTok{(pred_prey_data}\OperatorTok{$}\NormalTok{Prey.mass.unit) }\CommentTok{#shows now all in grams}
\end{Highlighting}
\end{Shaded}

\begin{verbatim}
## [1] "g"
\end{verbatim}

We have now got data on predator/prey masses we can work with for
comparisons.

\hypertarget{predator-and-prey-mass-data}{%
\subsection{Predator and Prey Mass
data}\label{predator-and-prey-mass-data}}

Now assign the data for Predator and Prey mass in grams to new data
frames:

\begin{Shaded}
\begin{Highlighting}[]
\NormalTok{pred_mass_g <-pred_prey_data}\OperatorTok{$}\NormalTok{Predator.mass }\CommentTok{#all in grams anyway}
\NormalTok{prey_mass_g <-pred_prey_data}\OperatorTok{$}\NormalTok{Prey.mass }\CommentTok{# now all converted to grams}
\end{Highlighting}
\end{Shaded}

We can view summaries of the data for Predator Mass and Prey Mass:

\begin{Shaded}
\begin{Highlighting}[]
\KeywordTok{summary}\NormalTok{(pred_mass_g)}
\end{Highlighting}
\end{Shaded}

\begin{verbatim}
##    Min. 1st Qu.  Median    Mean 3rd Qu.    Max. 
##       0     172    1751   15292    6326  415600
\end{verbatim}

\begin{Shaded}
\begin{Highlighting}[]
\KeywordTok{plot}\NormalTok{(pred_mass_g)}
\end{Highlighting}
\end{Shaded}

\includegraphics{Barnes-et-al-predator-prey-bodysizes-R-markdown_files/figure-latex/unnamed-chunk-9-1.pdf}

This shows that whilst the average predator has a mass of about 15,000g,
there are some predators that have a mass of up to 415,600g.

Now if we wanted to compare the mass of a predator with the mass of its
prey, we could do this using the ggplot function, which has more
features than the inbuilt plot function:

\begin{Shaded}
\begin{Highlighting}[]
\NormalTok{pred_prey_data }\OperatorTok
\StringTok{  }\KeywordTok{ggplot}\NormalTok{(}\KeywordTok{aes}\NormalTok{(}\DataTypeTok{x=}\NormalTok{ prey_mass_g, }\DataTypeTok{y=}\NormalTok{ pred_mass_g)) }\OperatorTok{+}
\StringTok{  }\KeywordTok{geom_point}\NormalTok{()}
\end{Highlighting}
\end{Shaded}

\includegraphics{Barnes-et-al-predator-prey-bodysizes-R-markdown_files/figure-latex/unnamed-chunk-10-1.pdf}

Its hard to draw conclusions from this graph because of the scale of the
graph - most values are clustered towards the lower end of the mass
scale. If we take the log of both sides, this should represent the data
in a better way visually:

\begin{Shaded}
\begin{Highlighting}[]
\NormalTok{pred_prey_data }\OperatorTok
\StringTok{  }\KeywordTok{ggplot}\NormalTok{(}\KeywordTok{aes}\NormalTok{(}\DataTypeTok{x=} \KeywordTok{log}\NormalTok{(prey_mass_g), }\DataTypeTok{y=} \KeywordTok{log}\NormalTok{(pred_mass_g))) }\OperatorTok{+}
\StringTok{  }\KeywordTok{geom_point}\NormalTok{(}\DataTypeTok{shape=}\DecValTok{20}\NormalTok{) }\OperatorTok{+}
\StringTok{  }\KeywordTok{ggtitle}\NormalTok{(}\StringTok{"Log-Log Plot of Predator-Prey Masses"}\NormalTok{) }\OperatorTok{+}
\StringTok{  }\KeywordTok{xlab}\NormalTok{(}\StringTok{"log(Prey Mass (g))"}\NormalTok{) }\OperatorTok{+}
\StringTok{  }\KeywordTok{ylab}\NormalTok{(}\StringTok{"log(Predator Mass (g))"}\NormalTok{)}
\end{Highlighting}
\end{Shaded}

\includegraphics{Barnes-et-al-predator-prey-bodysizes-R-markdown_files/figure-latex/unnamed-chunk-11-1.pdf}

(Note we have added titles and labels to the graph, and made the
scatterpoints smaller so they can be differentiated more easily). As we
can see from this graph, there appears to be a correlation between the
natural logarithm of predator mass and the natural logarithm of prey
mass. This implies that the masses follow a power law relationship, i.e
the function is of the form y=a*x\^{}b.

\end{document}
